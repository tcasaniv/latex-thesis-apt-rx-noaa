% Macros personalizados (similar a componentes en Astro.js)
% Estos comandos facilitan la reutilización de elementos comunes



% Niveles de secciones para plan de tesis y tesis final
% -1 \part{part}
\newcommand{\planpart}[1]{\ifplanthesis\part*{#1}\else\part{#1}\fi}
% 0 \chapter{chapter}
\newcommand{\planchapter}[1]{\ifplanthesis\chapter*{#1}\else\chapter{#1}\fi}
% 1 \section{section}
\newcommand{\plansection}[1]{\ifplanthesis\section*{#1}\else\section{#1}\fi}
% 2 \subsection{subsection}
\newcommand{\plansubsection}[1]{\ifplanthesis\subsection*{#1}\else\subsection{#1}\fi}
% 3 \subsubsection{subsubsection}
\newcommand{\plansubsubsection}[1]{\ifplanthesis\subsubsection*{#1}\else\subsubsection{#1}\fi}
% 4 \paragraph{paragraph}
\newcommand{\planparagraph}[1]{\ifplanthesis\paragraph*{#1}\else\paragraph{#1}\fi}
% 5 \subparagraph{subparagraph}
\newcommand{\plansubparagraph}[1]{\ifplanthesis\subparagraph*{#1}\else\subparagraph{#1}\fi}



% Macro para insertar nota al pie con referencia a tecnología
\newcommand{\tech}[2]{#1\footnote{#2}}

% Macro para destacar términos técnicos
\newcommand{\termino}[1]{\textbf{#1}}

% Macro para insertar URL como referencia
\newcommand{\urlref}[2]{\href{#1}{#2}\footnote{\url{#1}}}

% Macro para crear encabezados de sección con número y descripción
\newcommand{\seccionNumerada}[2]{
  \section{#1}
  \textit{#2}
}

% Macro para texto de código inline
\newcommand{\codigo}[1]{\texttt{#1}}

% Macro para notas importantes
\newcommand{\importante}[1]{
  \begin{tcolorbox}[colback=yellow!10,colframe=red!50!black,title=Importante]
    #1
  \end{tcolorbox}
}

% Macro para tips o consejos
\newcommand{\tip}[1]{
  \begin{tcolorbox}[colback=blue!10,colframe=blue!50!black,title=Tip]
    #1
  \end{tcolorbox}
}

% Macro para requisitos
\newcommand{\requisito}[2]{
  \textbf{REQ-#1:} #2
}

% Macro para historias de usuario
\newcommand{\historia}[3]{
  \begin{tcolorbox}[colback=green!10,colframe=green!50!black,title=Historia de Usuario]
    \textbf{ID:} #1 \\
    \textbf{Como} #2 \textbf{quiero} #3
  \end{tcolorbox}
}

% Macro para casos de uso
\newcommand{\casoUso}[3]{
  \begin{tcolorbox}[colback=purple!10,colframe=purple!50!black,title=Caso de Uso]
    \textbf{ID:} #1 \\
    \textbf{Nombre:} #2 \\
    \textbf{Descripción:} #3
  \end{tcolorbox}
}

% Macro para tecnologías usadas
\newcommand{\tecnologia}[3]{
  \begin{tabular}{|p{3cm}|p{9cm}|}
    \hline
    \textbf{#1} & #2 \\
    \hline
    \multicolumn{2}{|l|}{\small #3} \\
    \hline
  \end{tabular}
}

% Macro para la referencia rápida
\newcommand{\citar}[1]{\cite{#1}}

% Macro para insertar una figura con caption
\newcommand{\figura}[4]{
  \begin{figure}[H]
    \centering
    \includegraphics[width=#3\textwidth]{#1}
    \caption{#2}
    \label{fig:#4}
  \end{figure}
}

% Macro para arquitectura del sistema
\newcommand{\arquitectura}[2]{
  \begin{figure}[H]
    \centering
    \includegraphics[width=0.9\textwidth]{#1}
    \caption{Arquitectura del sistema: #2}
    \label{fig:arquitectura}
  \end{figure}
}

% Macro para pseudocódigo
\newcommand{\pseudo}[2]{
  \begin{algorithm}[H]
    \caption{#1}
    \label{alg:#2}
    #2
  \end{algorithm}
}

% Macro para ecuaciones con descripción
\newcommand{\ecuacion}[2]{
  \begin{equation}
    #1
    \label{eq:#2}
  \end{equation}
}

% Macro para comparativas
\newcommand{\comparativa}[1]{
  \begin{table}[H]
    \centering
    \caption{Comparativa de alternativas}
    \begin{tabularx}{\textwidth}{|X|X|X|X|}
      \hline
      \textbf{Característica} & \textbf{Opción 1} & \textbf{Opción 2} & \textbf{Opción 3} \\
      \hline
      #1
      \hline
    \end{tabularx}
  \end{table}
}

% Macro para definir acrónimos
\newcommand{\acronimo}[2]{
  \textsc{#1} (#2)
}