% % config/variables.tex

% Variables globales para el proyecto de tesis
% Esta estructura se asemeja a los archivos de datos en Astro.js
\planthesistrue % El documento cambia a modo Plan de Tesis
% \planthesisfalse % El documento cambia a modo Tesis

% Información del proyecto
\newcommand{\VarTituloTesis}{Diseño e Implementación de estación terrena automatizada de bajo coste para recepción de imágenes meteorológicas de satélites NOAA en Arequipa}
\newcommand{\VarAutorUno}{Gutierrez Flores, Miguel Abiel}
\newcommand{\VarAutorDos}{} % Dejar en blanco para un solo autor
\newcommand{\VarGrado}{Ingeniero en Telecomunicaciones}
\newcommand{\VarAsesor}{Dr. Jimenez Montes de Oca, Romel Luis}
\newcommand{\VarInstitucion}{UNIVERSIDAD NACIONAL DE SAN AGUSTÍN DE AREQUIPA}
\newcommand{\VarFacultad}{FACULTAD DE INGENIERÍA DE PRODUCCIÓN Y SERVICIOS}
\newcommand{\VarEscuela}{ESCUELA PROFESIONAL DE INGENIERÍA EN TELECOMUNICACIONES}
\newcommand{\VarFecha}{\today}
\newcommand{\VarCiudad}{Arequipa}
\newcommand{\VarPais}{Perú}
\newcommand{\VarAnio}{2025}

% Palabras clave
\newcommand{\VarPalabrasClave}{IoT, Prototipo, ESP8266, Dashboard}
\newcommand{\VarKeywords}{IoT, Prototipo, ESP8266, Dashboard}

% Resumen ejecutivo
\newcommand{\VarResumen}{
  El resumen debe comunicar claramente la idea principal del tema de investigación, enfocándose en los aspectos más relevantes del problema que se aborda, así como en la propuesta de solución correspondiente.
}
\newcommand{\VarAbstract}{
  The abstract should clearly communicate the main idea of the researchtopic, focusing on the most relevant aspects of the problem being addressed, as well as the corresponding solution proposal.
}
\newcommand{\VarDescripcionPDF}{Plantilla de tesis en LaTeX.}

% Objetivos
\newcommand{\VarObjetivoGeneral}{
Diseñar e implementar un prototipo funcional de sistema que integre dispositivos IoT, un dashboard web para visualización y control para...
}

\newcommand{\VarObjetivosEspecificos}{
\begin{itemize}
  \item Diseñar la arquitectura hardware y software del sistema.
  \item Desarrollar los dispositivos para monitoreo y control de elementos del sistema.
  \item Implementar una plataforma web que permita la visualización y control de todos los componentes.
  \item Evaluar el rendimiento, seguridad y usabilidad del sistema completo.
\end{itemize}
}

% Alcances y limitaciones
\newcommand{\VarAlcances}{
\begin{itemize}
  \item Desarrollo de dispositivos para ...
  \item Creación de una plataforma web responsiva con visualización en tiempo real.
  \item Implementación de ...
  \item Integración de todos los componentes en un sistema unificado.
\end{itemize}
}

\newcommand{\VarLimitaciones}{
\begin{itemize}
  \item El prototipo se limitará a un entorno controlado.
  \item El sistema tendrá un conjunto predefinido de comandos reconocibles.
  \item No se incluirá integración con plataformas comerciales existentes.
  \item El sistema no contemplará algoritmos de aprendizaje adaptativo.
\end{itemize}
}

% Cronograma
\newcommand{\VarDuracionProyecto}{17 semanas}
\newcommand{\VarFechaInicio}{Marzo de 2025}
\newcommand{\VarFechaFin}{Julio de 2025}


% Dedicatoria y Agradecimientos
\newcommand{\VarDedicatoria}{
  A Dios por siempre acompañarme y bendecirme para lograr cumplir mis objetivos. Y a mi
  familia por demostrar que puedo contar con ella cuando se presente cualquier adversidad
  y por brindarme su apoyo constante.

  A mi familia por haberme apoyado incondicionalmente en todo momento durante toda mi
  formación académica, en especial a mi madre que a pesar de su enfermedad siempre estuvo
  a mi lado.
}
\newcommand{\VarAgradecimientos}{
  Agradezco a Dios, por estar en cada día de mi vida, por fortalecerme y cuidarme tanto
  en lo personal y profesional; A la Universidad Nacional de San Agustín, mi alma mater,
  en especial a la plana de docentes de la Escuela Profesional de Ingeniería en 
  Telecomunicaciones.

  Agradezco a mi familia por su constante aliento y apoyo y especial reconocimiento a 
  mi asesor de tesis, por su apoyo, crítica y correcciones para mejorar el trabajo.
}
