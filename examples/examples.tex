% Ejemplos de uso de componentes para tesis de ingeniería
% Este archivo muestra cómo utilizar los componentes definidos

\chapter*{Ejemplos de uso de componentes LaTeX}

\section*{Componentes para código fuente}
A continuación se muestra cómo utilizar el componente \texttt{code-listing.tex} para incluir código fuente en tu tesis:
\input{components/code-listing}

\section*{Componentes para diagramas}
Para incluir diagramas UML, de flujo, o de arquitectura, utiliza el componente \texttt{diagram.tex}:

% components/diagram.tex
% Componente para diagramas de arquitectura
\newcommand{\ArchitectureDiagram}[3]{%
  % #1: Anchura (ej: 0.8\textwidth)
  % #2: Título o caption
  % #3: Ruta de la imagen
  \begin{figure}[ht]
    \centering
    \includegraphics[width=#1]{#3}
    \caption{#2}
    \label{fig:#2}
  \end{figure}
}

\section*{Componentes para algoritmos}
Para documentar algoritmos con pseudocódigo formal, utiliza el componente \texttt{algorithm.tex}:
\input{components/algorithm}

\section*{Componentes para evaluación}
Para presentar resultados de evaluación en formato tabular:
% components/evaluation-table.tex
% Componente para mostrar resultados de evaluación
\newcommand{\EvaluationTable}[2]{%
  % #1: Título o caption
  % #2: Comando con el contenido de la tabla
  \begin{table}[ht]
    \centering
    \caption{#1}
    \label{tab:#1}
    #2
  \end{table}
}

\section*{Componentes para arquitectura del sistema}
Para mostrar la arquitectura completa del sistema:
% Componente para mostrar la arquitectura del sistema
% Uso: \SystemArchitecture{título}{descripción}{ruta-imagen}{etiqueta}

\newcommand{\SystemArchitecture}[4]{
  \begin{figure}[H]
    \centering
    \includegraphics[width=0.95\textwidth]{#3}
    \caption{#1: #2}
    \label{fig:#4}
  \end{figure}
}

% Ejemplo de arquitectura creada directamente con TikZ
\begin{figure}[H]
  \centering
  \begin{tikzpicture}[node distance=2cm, auto]
    % Estilo para capas
    \tikzstyle{layer}=[rectangle, rounded corners, minimum width=3cm, minimum height=1cm, text centered, draw=black, fill=blue!20]
    \tikzstyle{device}=[rectangle, rounded corners, minimum width=2cm, minimum height=0.7cm, text centered, draw=black, fill=green!20]
    \tikzstyle{cloud}=[rectangle, rounded corners, minimum width=3cm, minimum height=1cm, text centered, draw=black, fill=gray!20]
  
  \end{tikzpicture}

\end{figure}

\section*{Componentes para requisitos}
Para documentar requisitos funcionales y no funcionales:
\input{components/requirement-table}

\section*{Componentes para prototipos}
Para mostrar imágenes o diagramas de prototipos:
\input{components/prototype-figure}

\section*{Componentes para casos de prueba}
Para documentar los casos de prueba utilizados:
\input{components/test-case}

\section*{Componentes para diagramas de Gantt}
Para mostrar la planificación temporal del proyecto:
\input{components/gantt-chart}

\section*{Componentes para documentación de API}
Para documentar las APIs desarrolladas:
\input{components/api-documentation}

\section*{Componentes para esquemas de bases de datos}
Para visualizar el modelo de datos:
\input{components/database-schema}

\section*{Componentes para mockups de interfaces}
Para mostrar diseños de interfaces de usuario:
\input{components/ui-mockup}

\section*{Componentes para diagramas de circuitos}
Para incluir esquemas electrónicos:
\input{components/circuit-diagram}

\section*{Componentes para derivaciones matemáticas}
Para incluir fórmulas matemáticas con explicaciones:
\input{components/math-derivation}









