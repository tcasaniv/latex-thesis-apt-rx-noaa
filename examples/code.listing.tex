% Componente para mostrar código fuente en la tesis
% Uso: \CodeListing{lenguaje}{título}{etiqueta}{ruta-archivo}

\newcommand{\CodeListing}[4]{
  \begin{figure}[H]
    \centering
    \begin{minipage}{0.95\textwidth}
      \lstinputlisting[
        language=#1,
        caption=#2,
        label=lst:#3
      ]{#4}
    \end{minipage}
    \caption*{Listado de código: #2}
  \end{figure}
}

% Uso directo en el documento
\begin{figure}[H]
  \centering
  \begin{minipage}{0.95\textwidth}
    \begin{lstlisting}[language=JavaScript, caption={Servicio de API para dashboard IoT}, label=lst:api-service]
// Servicio API para el dashboard IoT
import axios from 'axios';

export default class IoTService {
  constructor(baseURL) {
    this.api = axios.create({
      baseURL: baseURL || 'https://api.smart-home.example',
      timeout: 5000,
      headers: {
        'Content-Type': 'application/json',
        'Accept': 'application/json'
      }
    });
  }

  // Obtener estado actual de todos los dispositivos
  async getDeviceStatus() {
    try {
      const response = await this.api.get('/devices/status');
      return response.data;
    } catch (error) {
      console.error('Error al obtener estado de dispositivos:', error);
      throw error;
    }
  }

  // Controlar un dispositivo específico
  async controlDevice(deviceId, command) {
    try {
      const response = await this.api.post(`/devices/${deviceId}/control`, {
        command: command
      });
      return response.data;
    } catch (error) {
      console.error(`Error al controlar dispositivo ${deviceId}:`, error);
      throw error;
    }
  }
}
    \end{lstlisting}
  \end{minipage}
  \caption*{Ejemplo de código: Servicio API para Dashboard IoT}
\end{figure}

% Ejemplo de uso del comando \CodeListing:
% \CodeListing{Python}{Algoritmo de procesamiento de voz}{voice-processing}{assets/code/voice_processing.py}
